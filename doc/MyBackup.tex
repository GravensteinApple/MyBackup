\documentclass[a4paper]{article}

\usepackage[english]{babel}
\usepackage[utf8]{inputenc}
\usepackage{amsmath}
\usepackage{graphicx}
\usepackage[colorinlistoftodos]{todonotes}

\title{\textbf{MyBackup}}

\author{Leo Li}

\date{\today}

\begin{document}
\maketitle

\begin{abstract}
MyBackup is a small tool used to backup your files in linux.
\end{abstract}

\section{Introduction}

The safety of our files is always a big part of our concern. In mac OS, an efficient tool called Time Machine is used to backup your files which is amazingly easy to use.\\\\
        Therefore, I write a program named MyBackup to come up with a solution for your files' safety in Linux. This tool not only support incremental backup, but can automatically run at a specific time per day. In addition, you can check your backup status whenever you like and the UI is user-friendly as well.

        \section{Installation}

        1. unzip the package. \\
            2. cd in the unzipped folder \\
            3. use command \$ aclocal \\
            4. use command \$ autoconf \\
            5. use command \$ automake --add-missing \\
            6. use command \$ ./configure \\
            7. use cammand \$ sudo make install \\\\
            The executable program will be installed in /usr/local/bin/ \\
            The documentation will be installed in /usr/local/share/doc/MyBackup/ \\
            The configurtion files will be found in /usr/local/etc/MyBackup/

            \section{Uninstallation}

            1. cd in the original folder \\
                2. use command \$ sudo make uninstall\\
                3. then all the related files will gone

                \section{How to use}

                \subsection{Configuration}

                Move into /usr/local/etc/Mybackup/, then you will find two conf files, 
                one is named mybackup.conf which can be modified as your prefer,
                the other one named exclude.conf is a list of files you do not like to backup,
                one \textbf{relative path} per line.

                \subsection{Usage}

                \subsubsection{mybackup --help}

                Get the help page.

                \subsubsection{mybackup --push}

                Create a backup file of your "backup" to your "target".\\
                    Pay attention that if the number of the backups surpass the limit, the earliest one will be removed.

                    \subsubsection{mybackup --status}

                    You will see the backup status: \\
                        1. the time you create a backup\\
                        2. which is your last sync\\
                        3. how many backups you have\\
                        4. the status of the automactic backup

                        \subsubsection{mybackup --config}

                        You can see the configuration you set. To reset it, get into /usr/local/etc/MyBackup to modify the configuration files.

                        \subsubsection{mybackup --restore [backup\_name]}

                        You can use this command to restore the backup you want. Use\\
                            \$ mybackup --status\\
                            to find the backup name. \\\\
                            For example:\\
                            \$ mybackup --restore 2014-05-28T13\_36\_53

                            \subsubsection{mybackup --clean}

                            Clean all the backups

                            \subsubsection{mybackup --automatic [on:off]}

                            This command will let the automatically backup on or off.\\
                                To see the current automatic status, just type\\
                                \$ mybackup --status\\
                                To see the time you set, just type\\
                                \$ mybackup --config

                                \section{Contact me}
                                leo.liyunzhu@pku.edu.cn

                                \end{document}
